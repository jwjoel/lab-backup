%%%%%%%%%%%%%%%%
% Set options

\newcommand{\settitle}{Homework 2}
\newcommand{\course}{CSE 207B}
\newcommand{\coursename}{Applied Cryptography}
\newcommand{\assigndate}{October 3, 2024}
\newcommand{\duedate}{Tuesday, October 15}

\newcommand{\Ex}{\mathbb{E}}
\newcommand{\e}{\epsilon}
\newtheorem{theorem}{Theorem}

%%%%%%%%%%%%%%%%

\documentclass[letterpaper,12pt]{article}
\usepackage[top=1in, bottom=1in, left=1in, right=1in]{geometry}
\usepackage{fancyvrb}
\usepackage[protrusion=true,expansion=auto]{microtype}
\usepackage{color}
\usepackage[T1]{fontenc}
\usepackage{lmodern}
%\usepackage{mathptmx}
\usepackage{textcomp}
\usepackage[
  breaklinks=true,colorlinks=true,linkcolor=black,%
  citecolor=black,urlcolor=black,bookmarks=false,bookmarksopen=false,%
  pdfauthor={\course},%
  pdftitle={\settitle},%
  pdftex]{hyperref}
\usepackage{amsmath}
\usepackage{amsfonts}
\usepackage{multicol}
\renewcommand{\ttdefault}{cmtt}
\def\textsb#1{{\fontseries{sb}\selectfont #1}}

\newcommand{\problemsetdone}{\hfill$\Box{}$}

\newcommand{\htitle}
{
    \vbox to 0.25in{}
    \noindent\parbox{\textwidth}
    {
        \course\hfill \assigndate\newline
        \coursename\hfill 
        Due: \duedate \vspace*{-.5ex}\newline
        \mbox{}\hrulefill\mbox{}
    }
    \vspace{8pt}
    \begin{center}{\Large\bf{\settitle}}\end{center}
}
\newcommand{\handout}
{
    \thispagestyle{empty}
    \markboth{}{}
    \pagestyle{plain}
    \htitle
}

\newcommand{\problemsetheader}
{
Your homework should be submitted electronically via Gradescope before class on the due date.  Please type up your solutions to the following problems using Latex and submit in pdf form, along with text files containing the code you used to solve the first part.  Please credit any collaborators you worked with and any sources you used.

\medskip

\hrulefill

\bigskip
}

%%%%%%%%%%%%%%%%%%%%%%%%%%%%%%%%%%%%%%%%%
\newcommand{\E}{\operatorname{E}}
\newcommand{\Var}{\operatorname{Var}}
\newcommand{\Enc}{\operatorname{Enc}}
\newcommand{\Dec}{\operatorname{Dec}}

%%%%%%%%%%%%%%%%%%%%%%%%%%%%%%%%%%%%%%%%%

\begin{document}
\handout
\setlength{\parindent}{0pt}
\problemsetheader

\begin{enumerate}

\item[0.] Describe how you solved the decryption challenge.

\item Let $A$ be a fixed $m \times n$ matrix with $m > n$ whose entries are all binary.  Consider the following PRG $G : \{0,1\}^n \rightarrow \{0,1\}^m$ defined by
  \[
  G(s) := A \cdot s \bmod 2
  \]
  where $A \cdot s \bmod 2$ denotes a matrix-vector product where all elements of the resulting vector are reduced modulo 2.  Show that this PRG is insecure no matter what matrix $A$ is used. (Make sure your solution works modulo 2, not only for rationals.)

\item Let us study the security of a $4\Enc$ construction where a block cipher $(\Enc, \Dec)$ defined over a key space $K$ and a message space $X$ is iterated four times using four different keys: $4\Enc_{k_1,k_2,k_3,k_4}(m) := \Enc_{k_4}(\Enc_{k_3}(\Enc_{k_2}(\Enc_{k_1}(m))))$.

  \begin{enumerate}
  \item Show that there is a meet-in-the-middle attack on $4\Enc$ that recovers the secret key in time $|K|^2$ and memory space $|K|^2$.
    
  \item Suppose $|K| = |X|$. We will show, in the steps below, that there is a meet-in-the-middle attack on $4\Enc$ that recovers the secret key in time $|K|^2$, but only uses memory space $|K|$. If you get stuck, see ~\cite{doks12}.
      \begin{enumerate}
      \item Suppose we know, for some message-ciphertext pair $(m_1, c_1)$, the value in the middle of encrypting. That is, we know $X_1^2$ such that $$\Enc_{k_2}(\Enc_{k_1}(m_1)) = \Enc_{k_3}^{-1}(\Enc_{k_4}^{-1}(c_1)) = X_1^2$$
      Without looking at any other message-ciphertext pairs, how can we efficiently recover all possible values for $k_1$ and $k_2$? How many candidates would we expect there to be? 
      \item Say we use the above approach to narrow down our candidates for $k_1$ and $k_2$, then use a similar approach to narrow down our candidates for $k_3$ and $k_4$. If we are given a second message-ciphertext pair $c_2, m_2$, how can we efficiently reduce down to the remaining candidates for $(k_1, k_2, k_3, k_4)$?
      \item Using the above, describe an attack on $4\Enc$ that recovers the secret key in time $|K|^2$, but only uses memory space $|K|$, when $|K| = |X|$. 
      \end{enumerate}
  \end{enumerate}
\end{enumerate}
\vfill

\begin{thebibliography}{1}
\bibitem{doks12} I. Dinur, O. Dunkelman, N. Keller, and A. Shamir. Efficient dissection of composite problems, with applications to cryptanalysis, knapsacks, and combinatorial search problems. In
\textit{Advances in Cryptology--CRYPTO 2012}, pages 719--740. 2012.
  
\end{thebibliography}

\end{document}
